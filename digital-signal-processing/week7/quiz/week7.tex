\documentclass[a4paper]{article}

%% Language and font encodings
\usepackage[english]{babel}
\usepackage[utf8x]{inputenc}
\usepackage[T1]{fontenc}

%% Sets page size and margins
\usepackage[a4paper,top=3cm,bottom=2cm,left=3cm,right=3cm,marginparwidth=1.75cm]{geometry}

%% Useful packages
\usepackage{amsmath}
\usepackage{graphicx}
\usepackage[colorinlistoftodos]{todonotes}
\usepackage[colorlinks=true, allcolors=blue]{hyperref}
\usepackage{amsfonts}

\title{Digital Signal Processing - week7 quiz solutions}


\begin{document}
\maketitle

\section{Question 1}
Select the correct statement among the ones below:

\subsection{} 
\begin{enumerate}
\item A signal that is $\Omega_N$-bandlimited is also $\Omega_M$-bandlimited if $\Omega_M \ge \Omega_N$
\item This is obviously true
\end{enumerate}

\subsection{}
\begin{enumerate}
\item An interpolated discrete-time signal is always bandlimited.
\item false: for example pice-wise interpolation is not bandlimited 
\end{enumerate}

\subsection{}
\begin{enumerate}
\item The total energy of a continuous time signal is always infinite.
\item false, I think $\frac{1}{x}$ is a counter example since the integral of $\frac{1}{x^2}$ I think it's finite
\end{enumerate}

\subsection{}
\begin{enumerate}
\item In continuous time, the only measure of frequency is Hertz (i.e. 1/seconds)
\item false: in lecture 5.1 b they say it can also be measured in radians/sec
\end{enumerate}

\subsection{}
\begin{enumerate}
\item In continuous time, there is always a maximum frequency $\Omega$
\item nope
\end{enumerate}

\section{Question 2}
Easy to calc that it is $u(t+0.5) - u(t-0.5)$

\section{Question 4}
An important thing to note is that there are two similar formulas regarding the FT
of $rect$, one is with the frequencies in hertz , and one the frequency measure in rad/sec.
We are interested in the later.
This was solved in the homework excersice:

$$I(t) = 2rect(2t)\star rect(2t)$$
$$FT\{rect(t)\} = \frac{sin(\frac{\Omega}{2})}{\frac{\Omega}{2}} = sinc(\frac{\Omega}{2\pi})$$
$$ \Rightarrow I(j\Omega) = (2 * \frac{1}{2} \frac{sin(\frac{\Omega}{4})}{\frac{\Omega}{4}})^2 = \frac{1}{2}sinc(\frac{\Omega}{4\pi}) $$

\section{Question 5}
Was also solved in the homework

\section{Question 6}
When using a first-order interpolator (such as the one described in the previous question) to interpolate a finite-support sequence, which of the following statements are true?

\begin{enumerate}
\item The interpolated signal is bandlimited. - previous question shows that the interpolated signal
is a $sinc^2$ with intervals of 0 - so it is {\bf NOT} bandlimited

\item The interpolated signal has finite length in time because of the limited support of the interpolating function $I(t)$ - I think it's true, if the kernels are all finite support and the signal is finite support
then the interpolation must also be finite support. (Though it is not clear if a finite support also
means the the signal is finite or just that is infinite but is zero outside an interval

\item The spectrum between $[-\Omega_N,\Omega_N]$ (the baseband) is distorted by the non-flat response of the interpolating function over the baseband.- This is true, it is describe in quiestion 5(c) in the homework

\item The periodic copies of $X(e^{j \pi \frac{\Omega}{\Omega_N}})$ outside of $[-\Omega_N,\Omega_N]$
are not eliminated by the interpolation filter, since it is not an ideal lowpass. - Indeed and 
this is also shown in question 5 in the homework
\end{enumerate}


\section{Question 7}
Select the correct statement(s).
\begin{enumerate}
\item Increasing the interpolation interval, $T_S$ results in a wider spectrum of the interpolated signal. -
This is false, see lect 5.3.a
\item The sampling theorem implies that the space of bandlimited functions is a Hilbert Space.
- nope
\item The sampling of a bandlimited signal $x(t)$ (with maximum frequency $F_N$) with a sampling frequency $F_S \ge 2 \cdot F_N$ will result in no information loss. - This is it
\end{enumerate}

\section{Question 10}
This signal is $2\Omega_0$-bandlimited so it is enough to sample it in rate $T_s \le \frac{\pi}{2\Omega_0}$

\section{Question 11}
Assume $x(t)$ is a continuous-time pure sinusoid at 10 kHz. The signal is raw-sampled at 8 kHz and then interpolated back to a continuous-time signal with an interpolator at 8 kHz. What is the perceived frequency in kHz of the interpolated sinusoid?
\newline
\newline
We have $x(t) = sin(10^4 2\pi t)$ and we sample it every $\frac{1}{8*10^3}$ seconds so we have 
$$x[n] = x(\frac{1}{8*10^3}n) = sin(\frac{10^4 2\pi}{8*10^3}n) = sin(\frac{5}{4}2\pi n) = sin(2.5 \pi n) = sin(\frac{\pi}{2}n)$$
The DTFT of this $sin$ is two $\delta$ pulses at $\frac{\pi}{2}$ and $-\frac{\pi}{2}$.

From lecture 5.3.a we know that the spectrum of an interpolated signal is the same as it's DTFT but
scaled by $\frac{\Omega_N}{\pi}$. In our case $\Omega_N = 8*10^3*2\pi$ so the frequency of the interpolated signal is $\frac{\pi}{2} * \frac{\Omega_N}{\pi} = \frac{\Omega_N}{2} = 4*10^3*2\pi(rad/sec)$ and so in hertz it is  $4 * 10^3$ which is 4 KHz

\section{Question 13}
We can count 20 times in the plot

\section{Question 14-19}
TBD

\end{document}
